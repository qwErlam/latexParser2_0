\documentclass{article}
\usepackage[utf8]{inputenc}
\usepackage[russian]{babel}
\usepackage{setspace,amsmath}
\begin{document}
 
Говоря о графических методах, невозможно обойти одну проблему, "рождённую" практикой конкурсного экзамена. Мы имеем в виду вопрос о строгости , а следовательно, о законности решения, основанного на графических соображениях. Несомненно, с формальной точки зрения результат, снятый "с картинки", не подкреплённый аналитически, получен нестрого. Однако кем, когда и где определён уровень строгости, которого следует придерживаться абитурьенту? По нашему мнению, требования к уровню математической строгости для школьника должны определяться здравым смыслом. Более того, графический метод - всего лишь одно из средств наглядности. А наглядность может быть обманчивой. Так, для графиков функций $\displaystyle y = ({{1}\over{16}})^x$ и $\displaystyle y = log_{1/16}x$ "картинка", скорее всего покажет одну общую точку.
 
На самом деле их три. Таких примеров можно привести немало. Но, с нашей точки зрения, каждый из них свидетельствует только о том, что при использовании графических методов решения, впрочем как и аналитических, может быть допушена ошибка. Потому в тех случаях, когда результат, "прочитанный" с рисунка, вызвает сомнения, мы советуем подкрепить выводы  аналитически. Это следует сделать в первую очередь не для того, чтобы удовлетворить требования придирчивого  экзаменатора, а для подтверждения правоты выбранного пути решения. Таким образом, по отношению к школьнику "планка" математической строгости должна находиться в пределах разумной достаточности.
 
Надеемся, что мы не злоупротребим назидательностью, дав читателю ещё один совет. Настоящее пособие не преследует цель сформировать основу графической культуры, иными словами научить строить графики функций и уравнений. Чаще всего мы будем предлагать уже готовую картинку без описания процесса построения. Поэтому советуем читателю активизировать (если в этом есть необходимость) свои умения в построении графических образов, обратив особое внимание на построение графиков $ y = f(x+a),\; y = f(x) + a,\; y = f(|x|),\; y = |f(x)|,\; y = f(kx),\; y = kf(x)$ путём преобразований графика $ y = f(x)$. 
 
И последнее.Как нам кажется, в предлагаемых задачах ответ, полученный графически, требует затрат, по крайней мере, не больше, чем соответствующее аналитическое решение.
\\\\
\noindent \textbf{$\Lambda$. Параллельный перенос.)}
Начнем с задач, где членами семейства $y = f(x;a)$ будут прямые.
\\\\
\noindent \textbf{II.191.} (КГУ). Найти все значения параметра $b$, при которых уравнение
\\\\
$lg2|x| + lg(2-x) - lg(lgb) = 0$
\\\\
имеется единственное решение.
\\\\
Р\,е\,ш\,е\,н\,и\,е. Для удобства обозначим $lgb = a$. Запишем уравнение, равносильное исходному:
$lg (2|x|(2-x)) = lg\,a$. Переходим к равносильной системе:
\\\\
\begin{equation*}
 \begin{cases}
   2 |x|(2 - x) = a, 
   \\
   x < 2,
   \\
   x \ne 0.
 \end{cases}
\end{equation*}
\\\\
Строим график функции $2|x|(2-x)$ с область определения $x < 2$ и $x \ne 0$ (рис. 1). Полученный график семейство прямых $y = a$ должно пересекать только в одной точке. Из рисунка видно, что это требование выполняется лишь при $a > 2$, то есть $lg\,b > 2, b > 100$.
\\
О\,т\,в\,е\,т. $b > 100$.
\\\\
\noindent \textbf{II.192.(II.112.)} При каких значениях параметра $a$ неравенство $\sqrt{1 - x^2} > a - x$ имеет решения?
\\
Р\,е\,ш\,е\,н\,и\,е. Графиком функции $y =  \sqrt{1 - x^2}$ является полуокружность с центром $(0;0)$ и радиусом 1 (рис. 2).
\\
Функция $y = a - x$ для каждого фиксированного значения параметра задает прямую, то есть уравнение $y = a - x$ на координатной плоскости 
 
 
 
\end{document}