\documentclass[12pt]{article}
   \usepackage[utf8]{inputenc}
    \usepackage[russian]{babel} 
 
    \usepackage{graphicx}
 
    \title{Заголовок доклада}
    \author{авторы будут указаны позднее}
    \date{2014-2015}
 
    \begin{document}
    \maketitle
    \tableofcontents
    \newpage
    \begin{abstract}
         
    \end{abstract}
    \section{Постановка задачи}
    Обычный пятиугольник имеет пять углов и пять сторон.    
    На рис. (\ref{fig1}) изображен зеленый пятиугольник с красными ребрами.
    \begin{figure}[h]
    \centering
%   \includegraphics[scale=0.79]{image} % эту строку раскомментируйте если хотите работать с картинками
    \caption{Описание рисунка...}
    \label{fig1}
    \end{figure}
    Если сложить $2+2$, то к пятиугольнику на рис. (\ref{fig1}) ни что
    не мешает добавить еще два угла, и тем самым получить семиугольник.
    Вопрос заключается в том, насколько правильным получится новый
    семиугольник?
    \subsection{Правильные многоугольники}
    Правильность многоугольника существует тогда и только тогда,
    когда все его внутренние углы равны между собой. Этого достаточно
    чтобы многоугольник был правильным.
    \end{document}