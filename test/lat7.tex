\documentclass{article}
\usepackage[utf8]{inputenc}
\usepackage[russian]{babel}
\usepackage{setspace,amsmath}
\begin{document}
 
 
\textbf{II.55.} (МИЭМ). Решить уравнение $\displaystyle {{a - 3sinx}\over{acosx - 3}} = {{a - 3cosx}\over{asinx - 3}}$ и определить число его корней на отрезке $[49\pi; 49\pi]$.
 
\textbf{В. Параметр и свойства решений уравнений, неравенств и их систем.} Несколько слов скажем по поводу названия этого пункта. Оно во многом условно, и, вероятно, требует пояснений. Однако мы не будет делать это сейчас, надеясь, что в ходе непосредственной работы с задачами станет понятным, какие параметры попадают под этот пункт.
 
Начнём с задач, знакомым по главе I, в которых условие требует, чтобы ответ был каким-либо наперед заданным подмножествоммножества действительных чисел (см., например, \textbf{I.12}, \textbf{I.43}).
 
\textbf{II.56.} (ВГУ). Найти рациональные решения уравнения
$x + \sqrt{2} = x\sqrt{2} + a^2 $
где $a$ - рациональный параметр.
 
Р\,е\,ш\,е\,н\,и\,е.  Данное уравнение выгодно записать так:
$x - a^2  = \sqrt{2}(x - 1) $.
В силу условия левая часть этого уравнения принимает только рациональные значения. Тогда структура правой части позволяет делать вывод, что если исходное уравнение имеет рациональный корень, то он обязательно равен единице. Понятно, что при $x = 1$ получаем $a = \pm 1$. Легко установить справедливость обратного утверждения. Тогда
 
О\,т\,в\,е\,т. Если $a = \pm 1$, то $x = 1$; при других рациональных $a$ данное уравнение рациональных корней не имеет.
 
\textbf{II.57.} (МАИ). При каких значениях $a$ все решения уравнения $\displaystyle {{a - 1}\over{x + 6}} = {{2x + 7}\over{(x + 2)^2 - x - 22}}$ неположительные?
Р\,е\,ш\,е\,н\,и\,е.  Перепишем исходное уравнение в таком виде:
 $\displaystyle {{a - 1}\over{x + 6}} = {{2x + 7}\over{(x + 6)(x - 3)}}$.
 Переходим к равносильной системе:
 \begin{equation*}
 \begin{cases}
   x(a - 3) = 3a + 4, 
   \\
   x \ne -6,
   \\
   x \ne 3.
 \end{cases}
\end{equation*}
 
Ясно, что при $a = 3$ нет решений. Если $a \ne 3$, то $\displaystyle x = {{3a + 4}\over{a - 3}} $. Так как нас интересуют лишь неположительные решения, то искомые значения параметра $a$ найдём, решив систему
 \begin{equation*}
 \begin{cases}
  \displaystyle {{4 + 3a}\over{a - 3}} \leq 0, 
   \\
  \displaystyle  {{4 + 3a}\over{a - 3}} \ne -6,
 \end{cases}
\end{equation*}
О\,т\,в\,е\,т. $\displaystyle - {{4}\over{3}} \leq {{14}\over{9}}$ или $\displaystyle {{14}\over{9}} \leq 3$.
 
\textbf{II.58.} (МАИ). При каких $a$ уравнение
 
$\displaystyle    \bigg|{{(a - 1)x - (2a - 1)}\over{x - 1}}\bigg|  + \big|x - |1 - a|   + {{1}\over{2}}\big| = 0$  
имеет лишь положительные решения?
 
Р\,е\,ш\,е\,н\,и\,е. Так как левая часть данного уравнения  - сумма двух неотрицательных слагаемых, то это уравнение равносильно системе
 \begin{equation*}
 \begin{cases}
  \displaystyle {{(a - 1)x - (2a - 1)}\over{x - 1}} = 0 \leq 0, 
   \\
  \displaystyle  x - |1 - a| + {{1}\over{2}} = 0,
 \end{cases}
\end{equation*}
\\
Отсюда:
 \begin{equation*}
 \begin{cases}
  \displaystyle (a - 1)x = 2a - 1, 
   \\
  \displaystyle  x = |1 - a| - {{1}\over{2}},
  \\
  x \ne 1.
 \end{cases}
\end{equation*}
 
При $a = 1$ система не имеет решений. Если $a \ne 1$, то переходим к равносильной системе
 \begin{equation*}
 \begin{cases}
  \displaystyle x = {{2a - 1}\over{a - 1}}, 
   \\
  \displaystyle  x  = |1 - a| - {{1}\over{2}},
  \\
  x \ne 1.
 \end{cases}
\end{equation*}
 
Для того, чтобы эта система, а значит, и исходное уравнение имели положительные решения, достаточно потребовать
 
 
  
  
\end{document}