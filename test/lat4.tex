\documentclass{article}
\usepackage[utf8]{inputenc}
\usepackage[russian]{babel}
\usepackage{setspace}
\begin{document}
 
Утверждение 2.11 (свойства протокола Шамира).
 
\setcounter{section}{2}
\setcounter{subsection}{4}
\subsection{Шифр Эль-Гамаля}
 
Пусть именются абоненты $A, B, C, ...$, которые хотят передавать друго другу зашифрованные сообщения, не имея никаких защищённых каналов связи. В этом разделе мы рассмотрим шифр предложенный Эль-Намалем (Taher ElGamal), который решает эту задачу, используя, в отличии от шифра Шамира только одну пересылку сообщения. Фактически здесь используется схема Даффи-Хеллмана, чтобы сформировать общий секретный ключдля двух абонентов, передающих друг другу сообщение, и затем вообщение шифруется путём умножения его на этот ключ. Для каждого следующего сообщения секретный ключ вычисляется заново. Перейдём к точному описанию метода.
 
Для всей группы абонентов выбирается некотрое большое простое число $p$ и число $g$, такие что различные степени $g$ суть различные числа по модулю $p$ (см. раздел 2.2). Числа $p$ и $g$ передаются абонентам в открытом виде (они могут использоваться всеми абонентами сети).
 
Затем каждый абонент выбирает свое секретное число $c_i, 1$
 
\end{document}