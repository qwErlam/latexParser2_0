\documentclass{article}
\usepackage[utf8]{inputenc}
\usepackage[russian]{babel}
\usepackage{setspace,amsmath}
\begin{document}
 
 
Заметим, что тот, кто знаком с производной, может получить этот результат иначе.
 
Далее, смещая "полупараболу" влево, зафиксируем тот момент, когда графики $y = x + 1$ и $y = \sqrt{x + a} $ имеют две общие точки (положение III). Такое расположение обеспечивается требование ${{3}\over{4}} < a \leq 1 $.
 
Ясно, что в этом случаеотрезок $[x_1; x_2]$, где $x_1$ и $x_2$ - абсциссы точек пересечения графиков, будет решением исходного неравенства. Решив записанное выше уравнение, получим $\displaystyle x_1 = {{-1 - \sqrt{4a - 3}}\over{2}}, x_2 = {{-1 + \sqrt{4a - 3}}\over{2}} $. Следовательно, если ${{3}\over{4}} \leq a \leq 1$, то $ {{-1 - \sqrt{4a - 3}}\over{2}} \leq x \leq {{-1 + \sqrt{4a - 3}}\over{2}}$.
 
Когда "полупарабола" и прямая пересекаются только в одной точке (это соответсвует случаю $a > 1$), то решением будет отрезок $[-a; x_2]$, где $x_2$  - наибольший из корней $x_1$ и $x_2$ (положение IV).
Итак, мы привели два решенияэтой задачи. Третье решение читатель найдёт в \textsection 4.
\\\\
\textbf{II.196.} Сколько корней имеет уравнение $\sqrt{x+a} = log_{1/3}(x - 2a)$ в зависимости от значений параметра $a$?
  
 Р\,е\,ш\,е\,н\,и\,е. Заметим, что введя функции $y = \sqrt{x + a}$ и $y = log_{1/3}(x - 2a)$, мы получаем сразу два семейства кривых. В этом случае поиск общих точек затрудняется. Однако, задачу можно облегчить, применив замену $x -2a = t$. Отсюда получаем $\sqrt{t + 3a} = log_{1/3}t$. Рассмотрим функции $y = \sqrt{t + 3a}$ и $y = log_{1/3}t$.
Среди них лишь одна задаёт семейство кривых. Теперь мы видим, что произведенная замена приносит несомненную пользу. Параллельно отметим, что в предыдущей задаче аналогичной заменой можно заставить двигаться не "полупараболу", а прямую. Рекомендую читателю запомнить этот эффективный приём, ассоциирующийся с принципом относительности движений в кинематике.
 
Обратимся к рис. 6. Очевидно, что если абсцисса вершины "полупараболы" больше единицы, то есть
$-3a > 1, a < {{1}\over{3}}$, то уравнение корней не имеет. Если $a \geq - {{1}\over{3}}$, то по рисунку видно, что рассматриваемые графики пересекаются, причём только в одной точке, поскольку функции $y = \sqrt{t + 3a}$ и $y = log_{1/3}t$  имеют разный характер монотонности.
 
О\,т\,в\,е\,т. Если $a \geq - {{1}\sqrt{3}}$, то уравнение имеет один корень; если $a < - {{1}\sqrt{3}}$, то корней нет.
\\\\
\textbf{II.197.} (МВТУ). Найти все значения параметра $a$, при которых система уравнений
\begin{equation*}
 \begin{cases}
   x = a + \sqrt{y}, 
   \\
   y^2 - x^2 - 2x + 4y + 3 = 0
 \end{cases}
\end{equation*}
имеет решения.
 
Р\,е\,ш\,е\,н\,и\,е. Из первого уравнения системы получим $y = (x-a)^2$ при $x \geq a$. Следовательно, это уравнение опять-таки задаёт семейство "полупарабол" (параболы $y = (x - a)^2$ "скользят" вершинами по оси абсцисс, причём мы рассматриваем только правую ветвь).
 
Левую часть второго уравнения системы разложим на множители . Имеем $ y^2 - x^2 - 2x + 4y + 3 = (y^2 + 4y + 4) - (x^2 + 2x + 1) = (y + x + 3)(y - x + 1)$.
Тогда графиком второго уравнения является объединение двух прямых $y = -x - 3$ и $y = x - 1$. 
  
  
\end{document}